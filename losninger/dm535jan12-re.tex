\documentclass[a4paper,10pt]{article}

\usepackage[utf8]{inputenc}
\usepackage[T1]{fontenc}
\usepackage[english]{babel}

\usepackage{color}
\usepackage{float}
\usepackage{caption}
\usepackage{subcaption}
\usepackage{fancyvrb}
\newcommand{\N}{\mathbb{N}}
\newcommand{\R}{\mathbb{R}}
\newcommand{\Z}{\mathbb{Z}}
\usepackage{amssymb}
\usepackage{amsmath}
\usepackage{listings}

\usepackage{graphicx}
\DeclareGraphicsExtensions{.png}

\title{DM535 eksamenssæt -- 13. januar/12 \\ \rule{10cm}{0.5mm}}
\author{Studiegruppe F \\ Section S7 \\
DM535\\\rule{5.5cm}{0.5mm}\\}
\date{\today}

\begin{document}

\maketitle

\vfill

\tableofcontents

\newpage

\section{Bijektion, invers, $f+g$, $g \circ f$ og $f \circ g$}
\begin{align*}
f(x) &= x^2	+x+1\\
g(x) &= 2x-2
\end{align*}
\subsection*{Er $f(x)$ en bijektion?}
Funktionen $f(x)$ ikke er bijektiv, da funktionen ikke er injektiv og surjektiv.
\subsection*{Har $f(x)$ en invers funktion?}
Da funktionen $f(x)$ ikke er bijektiv, kan den heller ikke have en invers.
\subsection*{Beregn $f+g$}
\begin{align*}
(x^2+x+1)+(2x-2) = x^2+3x-1
\end{align*}
\subsection*{Bergen $f\circ g$}
\begin{align*}
(2x-2)^2+(2x-2)+1 = (4x^2+4-8x)+2x-1 = 4x^2-6x+3
\end{align*}
\subsection*{Beregn $g \circ f$}
\begin{align*}
2(x^2+x+1)-2 = (2x^2+2x+2)-2 = 2x^2+2x 
\end{align*}
\section{Udsagn $P$ og $Q$ og negering}
\begin{align*}
P:\exists x \in \N &: \forall y \in \N: x=y\\
Q:\forall x \in \N &: \exists y \in \N: x=y
\end{align*}
\subsection*{Er udsagn $P$ sandt?}
Udsagn $P$ er falsk, da udsagnet betyder, at vi for ét enkelt $x$ tilhørende de naturlige tal, skal kunne få alle $y$ tilhørende de naturlige tal. Hvis vi sætter $x = 1$ og $y\neq 1$ vil udsagnet ikke holde.
\subsection*{Er udsagn $Q$ sandt?}
Udsagn $Q$ er sandt, da udsagnet betyder, at der for alle $x$, tilhørende de naturlige tal, eksisterer mindst ét $y$ tilhørende de naturlige tal. Lad os tage et vilkårligt naturligt tal og kalde det vores $y$. Det vil her være muligt at sætte $x$ til at være netop det helt samme naturlige tal.
\subsection*{Negering af $P$}
\begin{align*}
P:\exists x \in \N &: \forall y \in \N: x=y\\
\neg P:\neg (\exists x \in \N &: \forall y \in \N: x=y)
\end{align*}
Ved brug af De Morgan's love for kvantorer, kan vi flytte negeringen forbi vores kvantorer. Så vi ved at $\neg (\exists x \in \N : \forall y \in \N: x=y)$ er ækvivalent med $\forall x \in \N : \neg (\forall y \in \N: x=y)$ som er ækvivalent med $\forall x \in \N : \exists y \in \N: \neg(x=y)$ som til sidst giver os, $\forall x \in \N : \exists y \in \N: x\neq y$.
\section{Talteori, indbyrdes primiske og kongruensen}
\subsection*{Hvilket par er indbyrdes primske?}
$(1)$ $15$ og $16\\
(2)$ $15$ og $20\\
(3)$ $15$ og $30$\\
For at finde det indbyrdes primske par, skal vi se hvilket par hvor kun $-1$ og $1$ går op i begge tal. Dette er ret nemt da f.eks. $5$ går op i begge tal i måde par nr. $2$ og par nr. $3$. Så det rigtige svar er par nr. $1$.\\\\
Tal der går op i 15 = $\{1,3,5,15\}$\\
Tal der går op i 16 = $\{1,2,4,8,16\}$
\subsection*{Angiv mindste positive heltal $x$ som opfylder kongruensen}
\begin{center}
$5x \equiv 1$ (mod $7$)
\end{center}
Det mindste heltal er $3$ da
\begin{align*}
5(1) \equiv 5 \,(mod \,7)\\
5(2) \equiv 3 \,(mod \,7)\\
5(3) \equiv 1 \,(mod \,7)
\end{align*}
\section{Binære relation}
\begin{align*}
R = \{(a,b)|b=a^2 \vee a = b^2 \}
\end{align*}
\subsection*{Angiv samtlige elementer i $R$}
Samtlige elementer i R: $\{(0,0),\,(1,1),\,(2,4),\,(3,9),\,(4,2),\, (9,3)\}$\\
\subsection*{Er $R$ refleksiv?}
For at R skulle være refleksiv, skal det gælde at: \begin{center}
$\forall a\in A : (a,a) \in R)$\\
\end{center} 
Dette er ikke tilfældet, da der mangler $\{(2,2)\}$\\
\subsection*{Er $R$ symetrisk?}
Hvis R skulle være symetrisk, skal det gælde at:
\begin{center}
$\forall a\in A : \forall b\in A : (a, b) \in R \Rightarrow (b, a) \in R$\\ 
\end{center}
Dette er tilfældet, da der i definationen på R, kun kan forekomme elementer, hvor $(a,b)$ enten er ens, eller hvor $(a,b)$ har en partner som er $(b,a)$.
\subsection*{Er $R$ transaktiv?)}
For at R er transaktiv, skal det gælde at:
\begin{center}
$\forall a\in A : \forall b \in A : \forall c \in A :  (a,b) \in R\, \wedge \, (b,c) \in R \Rightarrow (a,c) \in R
$
\end{center}
R er ikke transaktiv, da følgende elementer mangler: $\{(2,2)\}$\\R kunne også havde været transaktiv hvis elementerne $\{(4,2),\,(9,3)\} \vee \{(2,4,\,(3,9)\}$ ikke eksisterede.
\subsection*{Er $R$ en ækvivalensrelation?)}
For at R skulle være en ækvivalensrelation, skulle R både være refleksiv, symetrisk og transaktiv. Da R kun er symetrisk er dette ikke tilfældet.
\section{Matrice}
$A = \begin{bmatrix}
1&1\\1&1
\end{bmatrix}
$
\subsection*{Beregn $A^2$}
\begin{center}

$\begin{bmatrix}
(1\cdot1)+(1\cdot1)& (1\cdot1)+(1\cdot1)\\
(1\cdot1)+(1\cdot1)&(1\cdot1)+(1\cdot1)
\end{bmatrix}
$
$ A^2 =$ 
$\begin{bmatrix}
2&2\\2&2
\end{bmatrix}
$

\end{center}\newpage
\subsection*{b)}
Vis at $A^n \begin{bmatrix}
2^{n-1}&2^{n-1}\\2^{n-1}&2^{n-1}
\end{bmatrix}$\\\\
Matricen kalder vi for $P(n)$
\\\\Basisskridt:

$P(1):$\\\begin{center}
$A^1=\begin{bmatrix}
1&1\\1&1
\end{bmatrix}
$
\end{center}
Vi antager at $P(n)$ gælder og, at $P(n) \rightarrow P(n+1)$\\
$P(n+1)$
\begin{center}
$A^{n+1} = \begin{bmatrix}
2^n&2^n\\2^n&2^n
\end{bmatrix}$
\end{center}
Induktionsskridt:

Jeg vil nu vise at $P(n) \rightarrow P(n+1)$\\\begin{align*}
A^n\cdot A &= A^{n+1}\\
\begin{bmatrix}
2^{n-1}&2^{n-1}\\2^{n-1}&2^{n-1}
\end{bmatrix} \cdot \begin{bmatrix}
1&1\\1&1
\end{bmatrix} &= \begin{bmatrix}
2^n&2^n\\2^n&2^n\\
\end{bmatrix}\\
\begin{bmatrix}
2^{n-1}+2^{n-1}&2^{n-1}+2^{n-1}\\2^{n-1}+2^{n-1}&2^{n-1}+2^{n-1}
\end{bmatrix} &= \begin{bmatrix}
2^n&2^n\\2^n&2^n
\end{bmatrix}\\
\begin{bmatrix}
2\cdot 2^{n-1}&2\cdot 2^{n-1}\\2\cdot 2^{n-1}&2\cdot 2^{n-1}
\end{bmatrix}&= \begin{bmatrix}
2^n&2^n\\2^n&2^n
\end{bmatrix}\\
\begin{bmatrix}
2^n&2^n\\2^n&2^n
\end{bmatrix}&= \begin{bmatrix}
2^n&2^n\\2^n&2^n
\end{bmatrix}\\
\end{align*}
$\hfill\blacksquare
$
\end{document}
